%        File: 
%     Created: Sam Nov 03 06:00  2012 C
% Last Change: Sam Nov 03 06:00  2012 C
%
\documentclass[a4paper]{article}
\usepackage{amsmath}
\usepackage{amsthm}
\usepackage{amssymb}
\usepackage{multirow}
\numberwithin{equation}{section}
\title{Algorithmics WS13/14, Programming Exercise}
\author{Dino Rossegger \emph{0926471}\\ Markus Scherer \emph{1028046}}
\date{\today}
\begin{document}
\maketitle
\section{General}
A artificial root vertex $0$ was introduced in the graphs which edges going to all other nodes and the cost function $c_{0j}=0\quad \forall j\in V$. Since the formulations work with directed graphs for every undirected edge $e$, two new edges where introduced, $e_{ij}$ and $e_{ji}$ denoting the corresponding arcs.
For all Formulations the same objective function was used.
\begin{equation}
	\min{\sum_i \sum_{j\not=i} c_{ij}x_{ij}}
\end{equation} 
$c_{ij}$ is denoting the cost of the arc and $x_{ij}$ being $1$ if the arc is chosen or $0$ otherwise. 
Therefore a constraint limiting the values $x$ can take was used.
\begin{equation}
	x_{ij} \in \{0,1\} \qquad \forall (i,j)
\end{equation}
To tighten the constraints, an additional variable $v_i$ was introduced, being $1$ if vertex $i$ is selected and $0$ otherwise.
\begin{equation}
	v_i \in\{0,1\} \qquad \forall i=0,\dots,n
\end{equation}

\section{MTZ Formulation}
For the Miller-Tucker-Zemlin formulation an order $u_i$ is introduced, it ensures that there is no cycle.
\begin{equation}
x_{ij} + u_i \leq u_j + k*(1-x_{ij}) \qquad \forall (i,j)
\end{equation}
The start vertex must be the artificial vertex $0$.
\begin{equation}
	u_0=0
\end{equation}
There must be exactly one arc from the artificial vertex $0$ to any other vertex in the tree.
\begin{equation}
	\sum_{j\geq1} x_{0j} = 1
\end{equation}
$u_i$ must be between $0$ and $k$ since $k$ vertices are in the tree.
\begin{equation}
0\leq u_i \leq k \quad \forall i=0,\dots,n
\end{equation}
By the tree property $n=m+1$ the number of edges in the tree must be $k$.
\begin{equation}
	\sum_{(i,j)} x_{ij} = k
\end{equation}
Since there is an order, we are talking about arcs and for every vertex there must be at most one incoming arc.
\begin{equation}
	\sum_{i\geq 1} x_{ij}\leq 1 \qquad \forall j=1,\dots,n
\end{equation}
The number of vertices in the tree (excluding the start node) must be $k$.
\begin{equation}
	\sum_{i\geq 1}v_i=k
\end{equation}
Only $v_i \in k-MST$ must have $u_i>0$.
\begin{equation}
	u_i \leq k*v_i	\quad \forall i=1,\dots,n
\end{equation}
Whenever an arc is chosen, the incident vertices must have an order.
\begin{align}
	x_{ij} &\leq u_i \\
	x_{ij} &\leq u_j 
\end{align}
The sum of orders $u$ must be exactly $\frac{k(k+1)}{2}$.
\begin{equation}
	\sum_{j \geq 0} u_{j}=\frac{k(k+1)}{2} \\
\end{equation}
Whenever an arc is chosen, the corresponding arc in the other direction cannot be chosen.
\begin{equation}
	x_{ij} + x_{ij}\leq 1 \\
\end{equation}


\section{SCF Formulation}
A commodity is introduced. This commodity can flow along each 
arc and results thus in the following decision variables:
\begin{equation}
  f_{ij} \qquad \forall (i,j)
\end{equation}
There must be $k$ arcs in the tree (including arc from $0$)
\begin{equation}
	\sum_{(i,j)} x_{ij} = k
\end{equation}
The number of vertices in the tree must be $k$ ($0$ not included)
\begin{equation}
	\sum_{i \geq 1}v_i =k \\
\end{equation}
There must be exactly $1$ arc going out of $0$ in the solution
\begin{equation}
	\sum_{j \geq 1} x_{0j}=1\\
\end{equation}
If an arc is selected, vertices incident to it must be selected too
\begin{equation}
	x_{ij} \leq v_i\\
\end{equation}
\begin{equation}
	x_{ij} \leq v_j\\
\end{equation}
There can be either a forward or a backward arc in the solution, not both.
\begin{equation}
	v_i + x_{ij} + x_{ji} \leq v_j + 1\\
\end{equation}
The flow must be at most $k$, and at least $0$
\begin{equation}
	0 \leq f_{ij} \leq k\\
\end{equation}
Exactly $k$ units of flow are sent out from node $0$.
\begin{equation}
	\sum_{j>0} f_{0j}=k \\
\end{equation}
Every vertex in the solution consumes $1$ flow.
\begin{equation}
  \sum_{(i,j)} f_{ij} - \sum_{(j,i)} f_{ji} = v_j \qquad \forall j=1,\dots,n\\
\end{equation}
The flow going through an arc can be at most $k$.
\begin{equation}
	f_{ij}\leq kx_{ij}
\end{equation}

\section{MCF Formulation}
For every node a different commodity is introduced. These commodities can flow along each 
arc and result thus in the following decision variables:
\begin{equation}
  f_{ij}^l \in \{0,1\} \qquad \forall (i,j) \quad \forall l = 1,\dots, n\\
\end{equation}

Most of the constraints are the same as in the single commodity flow formulation.
\begin{align}
	\sum_{(i,j)} x_{ij} = k\\
	\sum_{i\geq1}v_i =k \\
	\sum_{j\geq1} x_{0j}=1\\
	\sum_{i\geq1} x_{i0}=0\\
	x_{ij} \leq v_i\\
	x_{ij} \leq v_j\\
	v_i + x_{ij} + x_{ji} \leq v_j + 1
\end{align}
The commodities of type $l$ going out of vertex $0$ must be exactly $v_l$ (only commodities of selected nodes are sent out) 
and the sum of all sent from $0$ commodities must be $k$.
\begin{equation}
	\sum_{j\geq1} f_{0j}^l = v_l \qquad \forall l= 1,\dots, n
\end{equation}
\begin{equation}
	\sum_{l=1}^n \sum_{(0,j)} f_{0j}^l = k\\
\end{equation}
The sum of flows on commodity $l$ flowing into vertex $l$ is at most $v_l$.
\begin{equation}
\sum_{i,i\not =l} f_{il}^l \leq v_l \qquad \forall l = 1,\dots,n\\
\end{equation}
If there is a flow on an arc, the arc must be selected.
\begin{equation}
0\leq f_{ij}^l \leq x_{ij} \qquad \forall (i,j) \quad \forall l = 0,\ldots,n
\end{equation}
All vertices which are not the target of the commodity can not consume a flow.
\begin{equation}
\sum_{i,i\not = j}f_{ij}^l - \sum_{i,i\not = j} f_{ji}^l =0 \qquad \forall j, l = 1,\dots,n \quad j\not =l \\
\end{equation}
$v_l$ cannot send out flow of it's own commodity.
\begin{equation}
\sum_{j\not=l} f_{lj}^l =0\\
\end{equation}
If there is flow incoming to vertex $l$ on its commodity, the vertex must be selected
\begin{equation}
\sum_{(i,l)} f_{il}^l - v_l=0 \qquad \forall l = 1,\dots,n \quad \forall i = 1,\dots,n\\
\end{equation}
The sum of incoming flows matching a node, over all commodities, must be $k$
\begin{equation}
	\sum_{l=1}^n \sum_{(i,l)} f_{i,l}^l = k\\
\end{equation}

%	\sum_{(0,j)} f_{0,j}^l -v_l = 0 \quad \forall l\in N\setminus\{0\}\\
If an arc is selected, there must be a flow on the arc.
\begin{equation}
x_{ij} \leq \sum_{l=1}^n f_{ij}^l
\end{equation}

\section{Results}

SCF is the only formulation that solves all instances. MTZ seems to have advantages over SCF with
smaller $k$-values but fails to solve the last configuration (after several hours the laptop
used for the benchmarks became unresponsive, possibly due overheating issues). MTZ also uses more
branch-and-bound-nodes.

Our MCF-Formulation is not able to solve all instances (for \emph{g06} it fails to prove optimality,
in the time we granted to it, although the optimal value is found, for \emph{g07} and \emph{g08} the
program crashes after exhausting the memory of the test computer (8GB)).


\begin{tabular}{ l l l l l l l l l}
  \multicolumn{2}{c}{} & \multicolumn{2}{c}{\textbf{MTZ}} & \multicolumn{2}{c}{\textbf{SCF}} & \multicolumn{2}{c}{\textbf{MCF}} & \textbf{Optimal Value}\\
                       & k   & time [s] & nodes & times [s] & nodes & time [s] & nodes &   \\
  \hline
  \multirow{2}{*}{g01} & 2   &          &       &           &       &          &       &   \\
                       & 5   &          &       &           &       &          &       &   \\
  \hline
  \multirow{2}{*}{g02} & 4   &          &       &           &       &          &       &   \\
                       & 10  &          &       &           &       &          &       &   \\
  \hline
  \multirow{2}{*}{g03} & 10  &          &       &           &       &          &       &   \\
                       & 25  &          &       &           &       &          &       &   \\
  \hline
  \multirow{2}{*}{g04} & 14  &          &       &           &       &          &       &   \\
                       & 35  &          &       &           &       &          &       &   \\
  \hline
  \multirow{2}{*}{g05} & 20  &          &       &           &       &          &       &   \\
                       & 50  &          &       &           &       &          &       &   \\
  \hline
  \multirow{2}{*}{g06} & 40  &          &       &           &       &          &       &   \\
                       & 100 &          &       &           &       &          &       &   \\
  \hline
  \multirow{2}{*}{g07} & 60  &          &       &           &       &          &       &   \\
                       & 150 &          &       &           &       &          &       &   \\
  \hline
  \multirow{2}{*}{g08} & 80  &          &       &           &       &          &       &   \\
                       & 200 &          &       &           &       &          &       &   \\
  \hline
\end{tabular}


\end{document}

