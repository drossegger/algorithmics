%        File: 
%     Created: Sam Nov 03 06:00  2012 C
% Last Change: Sam Nov 03 06:00  2012 C
%
\documentclass[a4paper]{article}
\usepackage{amsmath}
\usepackage{amsthm}
\usepackage{amssymb}
\usepackage{multirow}
\numberwithin{equation}{section}
\title{Algorithmics WS13/14, Programming Exercise}
\author{Dino Rossegger, Markus Scherer}
\date{\today}
\begin{document}
\maketitle
\section{General}
A artificial root vertex $0$ was introduced in the graphs which edges going to all other nodes and the cost function $c_{0j}=0\quad \forall j\in V$. Since the formulations work with directed graphs for every undirected edge $e$, two new edges where introduced, $e_{ij}$ and $e_{ji}$ denoting the corresponding arcs.
For all Formulations the same objective function was used. $c_{ij}$ denoting the cost function of the arc and $x_{ij}$ being $1$ if the arc is chosen or $0$ otherwise.
\begin{equation}
	\min{\sum_i \sum_{j\not=i} c_{ij}x_{ij}}
\end{equation} 
$c_{ij}$ denoting the cost function of the arc and $x_{ij}$ being $1$ if the arc is chosen or $0$ otherwise. Therefore a constraint limiting the values $x$ can take was used.
\begin{equation}
	x_{ij} \in \{0,1\} \qquad \forall (i,j)
\end{equation}
To tighten the constraints, an additional variable $v_i$ was introduced, being $1$ if vertex $i$ is selected and $0$ otherwise.
\begin{equation}
	v_i \in\{0,1\} \qquad \forall i=0,\dots,n
\end{equation}

\section{MTZ Formulation}
For the Miller-Tucker-Zemlin formulation an order $u_i$ is introduced, it ensures that there is no cycle.
\begin{equation}
x_{ij} + u_i \leq u_j + k*(1-x_{ij}) \qquad \forall (i,j)
\end{equation}
The start vertex must be the artificial vertex $0$.
\begin{equation}
	u_0=0
\end{equation}
There must be exactly one arc from the artificial vertex $0$ to any other vertex in the tree.
\begin{equation}
	\sum_{j>0} x_{0j} = 1
\end{equation}
$u_i$ must be between $0$ and $k$ since $k$ vertices are in the tree.
\begin{equation}
0\leq u_i \leq k \quad \forall i=0,\dots,n
\end{equation}
By the tree property $n=m+1$ the number of edges in the tree must be $k$.
\begin{equation}
	\sum_{(i,j)} x_{ij} = k
\end{equation}
Since there is an order, we are talking about arcs and for every vertex there must be at most one incoming arc.
\begin{equation}
	\sum_{i>0} x_{ij}\leq 1 \qquad \forall j=1,\dots,n
\end{equation}
The number of vertices in the tree must be $k$.
\begin{equation}
	\sum_{i>0}v_i=k
\end{equation}
Only $v_i \in k-MST$ must have $u_i>0$.
\begin{equation}
	u_i \leq k*v_i	\quad \forall i=1,\dots,n
\end{equation}
Whenever an arc is chosen, the incident vertices must have an order.
\begin{align}
	x_{ij} &\leq u_i \\
	x_{ij} &\leq u_j 
\end{align}
\section{SCF Formulation}
There must be $k$ arcs in the tree (including arc from $0$)
\begin{equation}
	\sum_{(i,j)} x_{ij} = k
\end{equation}
The number of vertices in the tree must be $k$ ($0$ not included)
\begin{equation}
	\sum_{i>0}v_i =k \\
\end{equation}
There must be exactly $1$ arc goint out of $0$ in the solution
\begin{equation}
	\sum_{j>0} x_{0j}=1\\
\end{equation}
If an arc is selected, vertices incident to it must be selected too
\begin{equation}
	x_{ij} \leq v_i\\
	x_{ij} \leq v_j\\
\end{equation}
There can be either a forward or a backward arc in the solution, not both.
\begin{equation}
	v_i + x_{ij} + x_{ji} \leq v_j + 1\\
\end{equation}
The flow must be at most $k$, and at least $0$
\begin{equation}
	0 \leq f_{ij} \leq k\\
	\sum_{j>0} f_{0j}=k \\
\end{equation}
Every vertex in the solution consumes $1$ flow.
\begin{equation}
	\sum_{(i,j)} f_{ij} - \sum_{(j,i)} f_{ji} = 1\\
\end{equation}
The flow going through an arc can be at most $k$.
\begin{equation}
	f_{i,j}\leq kx_{i,j}
\end{equation}

\section{MCF Formulation}
Most of the constraints are the same as in the single commodity flow formulation.
\begin{align}
	\sum_{(i,j)} x_{ij} = k\\
	\sum_{i>0}v_i =k \\
	\sum_{j>0} x_{0j}=1\\
	\sum_{i>0} x_{i0}=0\\
	x_{ij} \leq v_i\\
	x_{ij} \leq v_j\\
	v_i + x_{ij} + x_{ji} \leq v_j + 1\\
\end{align}
The commodities going out of vertex $0$ can be at most $1$ each and the sum of all commodities must be $k$.
\begin{equation}
	\sum_{j>0} f_{0j}^l \leq 1 \qquad \forall l\in [1,n]\\
	\sum_{l=1}^n \sum_{(0,j)} f_{0,j}^l = k\\
\end{equation}
The sum of flows on commodity $l$ flowing into vertex $l$ is at most $1$.
\begin{equation}
\sum_{i,i\not =l} f_{il}^l \leq 1 \qquad \forall l\in [1,n]\\
\end{equation}
If there is a flow on an arc, the arc must be selected.
\begin{equation}
0\leq f_{ij}^l \leq x_{i,j} \qquad \forall (i,j), \forall l \in [0,n]
\end{equation}
All vertices which are not the target of the commodity can not consume a flow.
\begin{equation}
\sum_{i,i\not = j}f_{ij}^l - \sum_{i,i\not = j} f_{ji}^l =0 \qquad \forall j, l\in [1,n], j\not =l \\
\end{equation}
If $l$ is target of some flow, the outgoing flow on its commodity must be $0$
\begin{equation}
\sum_{j\not=l} f_{l,j}^l =0\\
\end{equation}
If there is flow incoming to vertex $l$ on its commodity, the vertex must be selected
\begin{equation}
\sum_{(i,l)} f_{i,l}^l - v_l=0 \qquad \forall l\in [1,n], i\in[0,n]\\
\end{equation}
The sum of flows on all commodities must be $k$
\begin{equation}
	\sum_{l=1}^n \sum_{(i,l)} f_{i,l}^l = k\\
\end{equation}

%	\sum_{(0,j)} f_{0,j}^l -v_l = 0 \quad \forall l\in N\setminus\{0\}\\
If an arc is selected, there must be a flow on the arc.
\begin{equation}
x_{ij} \leq \sum_{l=1}^n f_{ij}^l
\end{equation}

\section{Results}

\begin{tabular}{ l l l l l l l l l}
  \multicolumn{2}{c}{} & \multicolumn{2}{c}{\textbf{MTZ}} & \multicolumn{2}{c}{\textbf{SCF}} & \multicolumn{2}{c}{\textbf{MCF}} & \textbf{Optimal Value}\\
                       & k   & time [s] & nodes & times [s] & nodes & time [s] & nodes &   \\
  \hline
  \multirow{2}{*}{g01} & 2   &          &       &           &       &          &       &   \\
                       & 5   &          &       &           &       &          &       &   \\
  \hline
  \multirow{2}{*}{g02} & 4   &          &       &           &       &          &       &   \\
                       & 10  &          &       &           &       &          &       &   \\
  \hline
  \multirow{2}{*}{g03} & 10  &          &       &           &       &          &       &   \\
                       & 25  &          &       &           &       &          &       &   \\
  \hline
  \multirow{2}{*}{g04} & 14  &          &       &           &       &          &       &   \\
                       & 35  &          &       &           &       &          &       &   \\
  \hline
  \multirow{2}{*}{g05} & 20  &          &       &           &       &          &       &   \\
                       & 50  &          &       &           &       &          &       &   \\
  \hline
  \multirow{2}{*}{g06} & 40  &          &       &           &       &          &       &   \\
                       & 100 &          &       &           &       &          &       &   \\
  \hline
  \multirow{2}{*}{g07} & 60  &          &       &           &       &          &       &   \\
                       & 150 &          &       &           &       &          &       &   \\
  \hline
  \multirow{2}{*}{g08} & 80  &          &       &           &       &          &       &   \\
                       & 200 &          &       &           &       &          &       &   \\
  \hline
\end{tabular}


\end{document}

