%        File: 
%     Created: Sam Nov 03 06:00  2012 C
% Last Change: Sam Nov 03 06:00  2012 C
%
\documentclass[a4paper]{article}
\usepackage{amsmath}
\usepackage{amsthm}
\usepackage{amssymb}
\numberwithin{equation}{section}
\title{Algorithmics WS13/14, Programming Exercise}
\author{Dino Rossegger, Markus Scherer}
\date{\today}
\begin{document}
\maketitle
\section{MTZ Formulation}
Let $x_{ij}=1$ if $e_{ij}\in E$ is in the Spanning Tree, else $x_{ij}=0$. Since the graph is undirected, for every edge there are to $x$, $x_{ij}$ and $x_{ji}$. The objective function is the same as in the other formulations
\begin{equation}
	\min{\sum_i \sum_{j\not=i} c_{ij}x_{ij}}
\end{equation}
For the Miller-Tucker-Zemlin formulation an order $u_i$ is introduced, it ensures that there is no cycle.
\begin{equation}
x_{ij} + u_i \leq u_j + k*(1-x_{ij}) \quad \forall i,j=0,\dots,n
\end{equation}
The start vertex must be the artificial vertex $0$.
\begin{equation}
	u_0=0
\end{equation}
There must be exactly one arc from the artificial vertex $0$ to any other vertex in the tree.
\begin{equation}
	\sum_{j>0} x_{0j} = 1
\end{equation}
$u_i$ must be between $0$ and $k$ since $k$ vertices are in the tree.
\begin{equation}
0\leq u_i \leq k \quad \forall i=0,\dots,n
\end{equation}
By the tree property $n=m+1$ the number of edges in the tree must be $k$.
\begin{equation}
	\sum_{i\geq0} \sum_{j\geq0,j\not =i} x_{ij} = k
\end{equation}
Since there is an order, we are talking about arcs and for every vertex there must be at most one incoming arc.
\begin{equation}
	\sum_{i>0} x_{ij}\leq 1 \quad \forall j=1,\dots,n
\end{equation}
Let $v_i=1$ if vertex $i \in k-MST$ and $0$ otherwise.
The number of vertices in the tree must be $k$.
\begin{equation}
	\sum_{i>0}v_i=k
\end{equation}
Only $v_i \in k-MST$ must have $u_i>0$.
\begin{equation}
	u_i \leq k*v_i	\quad \forall i=1,\dots,n
\end{equation}
Whenever an arc is chosen, the incident vertices must have an order.
\begin{align}
	x_{ij} &\leq u_i \\
	x_{ij} &\leq u_j 
\end{align}
\section{SCF Formulation}
For every edge $e \in E$ two arcs are introduced. $x_ij$ is $1$ if it is in the MST, $0$ otherwise.
\begin{equation}
	\min \sum_{i>0}{ \sum_{j>0,j\not =i}{x_{ij}c_{ij}}}
\end{equation}
\begin{align}
	\sum_{(i,j)} x_{ij} = k\\
	\sum_{i>0}v_i =k \\
	\sum_{j>0} x_{0j}=1\\
	x_{ij} \leq v_i\\
	x_{ij} \leq v_j\\
	v_i + x_{ij} + x_{ji} \leq v_j + 1\\
	0 \leq f_{ij} \leq k\\
	\sum_{j>0} f_{0j}=k \\
	\sum_{(i,j)} f_{ij} - \sum_{(j,i)} f_{ji} = 1\\
	f_{i,j}\leq kx_{i,j}
\end{align}

\section{MCF Formulation}
\begin{equation}
	\min \sum_{i>0}{ \sum_{j>0,j\not =i}{x_{ij}c_{ij}}}
\end{equation}
\begin{align}
	\sum_{(i,j)} x_{ij} = k\\
	\sum_{i>0}v_i =k \\
	\sum_{j>0} x_{0j}=1\\
	\sum_{i>0} x_{i0}=0\\
	x_{ij} \leq v_i\\
	x_{ij} \leq v_j\\
	v_i + x_{ij} + x_{ji} \leq v_j + 1\\
	\sum_{j>0} f_{0j}^l \leq 1 \quad \forall l\in N\setminus\{0\}\\
	\sum_{i,i\not =l} f_{il}^l \leq 1 \quad \forall l\in N\setminus\{0\}\\
	0\leq f_{ij}^l \leq x_{i,j}\\
	\sum_{i,i\not = j}f_{ij}^l - \sum_{i,i\not = j} f_{ji}^l =0 \quad \forall j, l\in N\setminus\{0\}, j\not =l \\
	\sum_{j\not=l} f_{l,j}^l =0\\
	\sum_{(i,l)} f_{i,l}^l - v_l=0 \quad \forall l\in N\setminus\{0\} \\
	\sum_{l=1}^N \sum_{(i,l)} f_{i,l}^l = k\\
	\sum_{l=1}^N \sum_{(0,j)} f_{0,j}^l = k\\
	\sum_{(0,j)} f_{0,j}^l -v_l = 0 \quad \forall l\in N\setminus\{0\}\\
	x_{ij} \leq \sum_{l=1}^N f_{ij}^l
\end{align}

\end{document}

